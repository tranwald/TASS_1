% Preambuła
\documentclass[a4paper,10pt]{article}
\usepackage[polish]{babel}
\usepackage[OT4]{fontenc}
\usepackage[utf8]{inputenc}

% Część główna

\title{Techniki Analizy Sieci Społecznych (TASS)\\Projekt 1}
\author{Andrzej Smyk}
\date{2 kwietnia 2015}
\begin{document}
	\maketitle
	\section{Temat projektu}

	Dopasowywanie sekwencji za pomocą algorytmów Gotoha oraz \linebreak\mbox{Altschula - Ericksona}. Porównanie z inną dostępną implementacją (np. z pakietem Biostrings w języku R).

	\section{Cel projektu}

	

	\section{Funkcjonalność rozwiązania}

	

	\section{Dopasowanie par sekwencji}

	ują one dodatkowo dwie macierze $I$ oraz $J$, złożoność pamięciowa jest trzykrotnie wyższa w porównaniu do algorytmów stosujących liniową funkcję kary.

	\section{Testowanie}

	Testowanie gotowego rozwiązania będzie składało się z dwóch części:
	\begin{enumerate}
		\item Testów wydajnościowych: ich celem będzie sprawdzenie czasów wykonania obu algorytmów dla sekwencji o różnej długości. Uzyskane czasy pozwolą na empiryczne oszacowanie złożoności czasowej algorytmów oraz porównanie ich z czasami wykonania dostępnych implementacji.
		\item Testów poprawnościowych: ich celem będzie weryfikacja, czy oba algorytmy prawidłowo dopasowują obie sekwencje tj. wykrywają zarówno mutacje, jak i insercje oraz delecje. W tym celu przetestujemy ich działanie na parach sekwencji, w których druga będzie zmodyfikowaną o mutacje oraz delecje i insercje wersją pierwszej.
	\end{enumerate}

\end{document}
